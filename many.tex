\documentclass[aps,prd,reprint]{revtex4-1}

\usepackage{hyperref}
\usepackage{amsmath}
\usepackage{amssymb}

\newcommand{\order}[1]{\mathcal{O}\left( #1 \right)}
\newcommand{\fixme}[1]{\textbf{FIXME: #1}}

\begin{document}

\title{Bayesian Rate Estimation}

\author{Will M. Farr}
\email{w-farr@northwestern.edu}
%\homepage{http://faculty.wcas.northwestern.edu/will-farr/}
% \affiliation{Center for Interdisciplinary Exploration and Research in Astrophysics (CIERA)\\
% Department of Physics and Astronomy\\
% Northwestern University, 2145 Sheridan Road, Evanston, IL 60208}
\affiliation{Northwestern University and CIERA}

\author{Ilya Mandel}
\email{ilya@chgk.info}
%\homepage{http://www.chgk.info/~ilyamandel/}
% \affiliation{School of Physics and Astronomy\\University of Birmingham\\Edgbaston B15 2TT Birmingham\\United Kingdom}
\affiliation{University of Birmingham}

\author{Jon Gair}

\begin{abstract}
  We show how to obtain a Bayesian estimate of the rate of signal
  events from a set of signal and background events indexed by a
  ranking statistic when the shapes of the signal and background
  distributions are known, can be estimated, or approximated.  We
  focus on the specific application of estimating astrophysical rates
  of the coalescence of compact binary black holes or neutron stars
  from a set of triggers in the LIGO/Virgo gravitational wave
  detectors, but our framework is fully general.  We discuss the
  systematic effects on the rate estimate due to differences between
  the assumed and true shapes of the foreground or background
  distributions, identifying ways these effects can be minimized.
  Similarly, we discuss the effects of various priors on the rate,
  including uninformative priors, weakly-informative priors, and the
  use of priors from previous rate experiments.  In the limit where
  the expected signal rate gives high probability of zero or one
  signal in the data, our technique reduces to the ``loudest event
  statistic,'' but it is generally applicable to arbitrarily large
  signal rates.
\end{abstract}

\maketitle

\section{Introduction}

\fixme{Introduce the necessity of estimating rates, prior work (like
  \cite{Biswas2009}), Bayesian inference.}

\section{Model}

We assume that we are presented with a data set of $N$ events.  Each
event may be due to either a signal of interest or an uninteresting
background.  Each event is associated with a ranking statistic, $x$.
Our data set therefore consists of the ranking statistics for the set
of events:
\begin{equation}
  d = \{ x_i | i = 1, \ldots, N \}.
\end{equation}
We assume that the events are sorted by ranking statistic, so that $i
< j$ implies that $x_i < x_j$.  

We assume that both the foreground and background events are samples
from an inhomogeneous Poisson process with rates (per unit ranking
statistic, $x$)
\begin{equation}
  \frac{dN_f}{dx} = f(x)
\end{equation}
and 
\begin{equation}
  \frac{dN_b}{dx} = b(x).
\end{equation}
The cumulative rates of the two processes are therefore
\begin{equation}
  F(x) \equiv \int_{-\infty}^x ds\, f(s)
\end{equation}
and
\begin{equation}
  B(x) \equiv \int_{-\infty}^x ds\, b(s).
\end{equation}
The assumption that the foreground and background events form an
inhomogeneous Poisson process implies
\begin{enumerate}
\item The number of events in any range of ranking statistics, $x \in
  [x_1, x_2]$ is Poisson distributed with rate $F(x_2) - F(x_1)$ or
  $B(x_2) - B(x_1)$.
\item The numbers of events in non-overlapping ranges of ranking
  statistics are independent. 
\item The probability of exactly one foreground event between $x$ and
  $x+h$ is given by
  \begin{equation}
    P(N = 1 \in [x, x+h]) = f(x) h + \order{h^2}.
  \end{equation}
  and similarly for background events.
\item The probability of two or more events in a small range of
  ranking statistic is negligable
  \begin{equation}
    P(N = 2 \in [x, x+h]) = \order{h^2}.
  \end{equation}
\end{enumerate}
The foreground and background rates can in general depend on several
parameters; the goal of our analysis is to determine the posterior
probability distributions for these parameters that are implied by the
data.  At the least, we will want to know the overall amplitude of the
foreground and background rates.  Let
\begin{equation}
  f(x) = R_f \hat{f}(x),
\end{equation}
and 
\begin{equation}
  b(x) = R_b \hat{b}(x),
\end{equation}
where $\hat{F}(\infty) = \hat{B}(\infty) = 1$.  Then $R_f$ and $R_b$
are the total number of foreground and background events expected,
respectively.  Other parameters may describe the shape of the rate
functions, but these will depend on the details of the system being
analyzed.  In \S \ref{sec:GW-example} we give an example of fitting
such shape paremeters.

We do not know a priori which of the events are foreground and which
are background.  For each event, we introduce a flag, $s_i$, which is
either 0 or 1, indicating a background or foreground event,
respectively.  These ``state'' flags are additional parameters in our
model.  We can marginalize over our uncertainty in the state of any
given event by summing all posteriors over $s_i = \{0,1\}$.  Given an
identification of each event as foreground or background, and the
rates of each, the likelihood of our data is
\begin{multline}
  p\left( d | \left\{ f(x), b(x), \left\{ s_i \right\} \right\}
  \right) = \\ \left[ \prod_{\left\{ i | s_i = 1 \right\}} f\left( x_i
    \right) \right]  \left[ \prod_{\left\{ j | s_j = 0 \right\}}
    b\left( x_j \right) \right] \\
  \times \exp\left[ - F\left( \infty \right) \right] 
  \exp\left[ -B\left( \infty \right) \right].
\end{multline}
Written in terms of the rate parameters, this becomes
\begin{multline}
  p\left( d | \left\{ f(x), b(x), \left\{ s_i \right\} \right\}
  \right) = \\ R_f^{N_f} \left[ \prod_{\left\{ i | s_i = 1 \right\}}
    \hat{f}\left( x_i \right) \right] \exp\left[ - R_f \right] \\
  \times R_b^{N_b} \left[ \prod_{\left\{ j | s_j = 0 \right\}}
    \hat{b}\left( x_j \right) \right] \exp\left[ - R_b \right],
\end{multline}
where $N_f$ is the number of the $s_i$ that are 1 (i.e.\ the number of
assumed foreground events), and $N_b$ is the number of the $s_i$ that
are 0 (i.e.\ the assumed number of background events).  Note that $N_f
+ N_b = N$, as each event is considered either foreground or
background in our model.

\subsection{Priors}

\fixme{Discussion of the various priors one might want to impose on
  the rates, $R_f$ and $R_b$, such as a flat prior on rate, the
  Poisson Jeffrey's prior (which is $\propto 1/\sqrt{R}$?), .}

A choice of the foreground and background rates implies a prior on the
state flags.  For fixed foreground and background rates, the
probability that any particular event is foreground is given by 
\begin{equation}
  p\left(s_i = 1 | R_f, R_b \right) = \frac{R_f}{R_f + R_b}.
\end{equation}

The complete prior is then
\begin{multline}
  p\left( R_f, R_b, \left\{ s_i \right\} \right) = \\ p\left( \left\{ s_i
  \right\} | R_f, R_b \right) p\left( R_f, R_b \right) = \\ \frac{R_f^{N_f} R_b^{N_b}}{\left( R_f
    + R_b\right)^N} p\left( R_f, R_b \right).
\end{multline}

\section{Gravitational Waves from Compact Binary Inspirals}
\label{sec:GW-example}

\begin{acknowledgments}
  Richard O'Shaughnessy for discussions.
\end{acknowledgments}

\appendix

\section{Likelihood for Inhomogeneous Poisson Processes}

Consider a set of samples, $\{ x_i \}$, from an 

%%%%%%%%%%%%%%%%%%%% Bibliography 
\bibliographystyle{apsrev4-1}
\bibliography{many}

\end{document}